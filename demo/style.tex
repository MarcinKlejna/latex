\documentclass[UTF8,10pt, twoside]{book}
\usepackage{ctexcap}   %中文支持 
%页面纸张设置
\usepackage{geometry}
\geometry{a4paper, papersize={210mm,297mm}, scale=0.8} %A4
\geometry{left=2cm,right=2cm,top=2cm,bottom=2cm}

% 设置中文段落首行缩进 
\usepackage{indentfirst}
\setlength{\parindent}{2em }  
% 设置行间距
\renewcommand{\baselinestretch }{1.2}
%设置段落间距
\addtolength{\parskip}{.4em}

%章节样式
\usepackage{titlesec}    
\titleformat{\chapter}{\centering\Huge\bfseries}{\chaptername}{1em}{}
\renewcommand{\chaptername}{第~\thechapter~章}
\titleformat{\section}{\LARGE\bfseries}{\thesection}{1em}{}
\titlespacing{\chapter}{0pt}{-20pt}{25pt}


%目录样式  
\CTEXsetup[name={第,章}, number={\arabic{chapter}}]{chapter} 
 

%页眉页脚
\usepackage{fancyhdr}
\pagestyle{fancy}
\fancyhf{}  
\renewcommand{\chaptermark}[1]{\markboth{第\thechapter 章\ ~~#1}{}}
\renewcommand{\sectionmark}[1]{\markright{\thesection ~~#1}{}}
\fancyhead[LO]{\leftmark}    %奇数页左侧章标题
\fancyhead[RE]{\rightmark}  %偶数页右侧节标题
\fancyhead[RO,LE]{-\,\thepage\,-}   %奇数页右侧, 偶数页左侧是页码
\renewcommand{\headrulewidth}{0pt} %无页眉线


%在目录中加入参考文献,索引等,   nottoc 不显示目录本身
\usepackage[nottoc]{tocbibind}

\usepackage{microtype}  %让字体更好看
\usepackage[colorlinks=false, pdfborder={0 0 0}]{hyperref}   %生成引用(图片/表格/公式)超链接

%引入图片,子图宏包
\usepackage{graphicx}
\usepackage{subfigure}
\setcounter{totalnumber}{2} %阻止 Latex将多于两个的浮动对象放置到同一页中
\usepackage{float}   %用于控制浮动
\usepackage[format=hang, font=small, textfont=it]{caption}
\DeclareCaptionLabelSeparator{twospace}{\ ~}
\captionsetup{labelsep=twospace}    %把图注序号与文本之间的冒号分隔符换成两个空格

%直立积分号
\usepackage{amsmath,amssymb}
\DeclareSymbolFont{EulerExtension}{U}{euex}{m}{n}
\DeclareMathSymbol{\euintop}{\mathop} {EulerExtension}{"52}
\DeclareMathSymbol{\euointop}{\mathop} {EulerExtension}{"48}
\let\intop\euintop
\let\ointop\euointop

%脚注样式
\usepackage{pifont}
\usepackage[perpage]{footmisc}  %每页脚注重新编号
\renewcommand{\thefootnote}{\ding{\numexpr191+\value{footnote}}}

%表格样式
\usepackage{multicol} 
\usepackage{multirow}
\usepackage{ctexcap}   
\usepackage{booktabs} 
\usepackage{colortbl}
\definecolor{tabcolor}{rgb}{.105,.410,.113}

%定理样式
\usepackage[T1]{fontenc}
\usepackage[utf8]{inputenc}
\usepackage{amsmath}
\usepackage{boiboites}  
\usepackage{chngcntr}
\newcounter{theoremcounter} 
\counterwithin{theoremcounter}{chapter} 
\newboxedtheorem[
boxcolor=orange,background=blue!5,titlebackground=blue!20,titleboxcolor=black]
{theo}{Theorem}{theoremcounter}


%伪代码/算法
\usepackage[vlined,ruled]{algorithm2e}


%绘图设定
%\usepackage{mathpazo}
\usepackage{tikz} 
\usepackage{pgfplots}
\pgfplotsset{compat=1.8}
\usetikzlibrary{arrows,intersections} 
% TikZ 设定
\tikzset{thick, >=stealth', dot/.style={draw,fill=white,circle,inner sep=0pt,minimum size=4pt}}


%流程图样式 
\usepackage{tikz}
\usetikzlibrary{shapes.geometric, arrows}
 \tikzstyle{startstop} = [rectangle, rounded corners, minimum width=3cm, minimum height=1cm,text centered, draw=black, fill=red!30]
 \tikzstyle{io} = [trapezium, trapezium left angle=70, trapezium right angle=110, minimum width=3cm, minimum height=1cm, text centered, draw=black, fill=blue!30]
 \tikzstyle{process} = [rectangle, minimum width=3cm, minimum height=1cm, text centered, draw=black, fill=orange!30]
 \tikzstyle{decision} = [diamond, minimum width=3cm, minimum height=1cm, text centered, draw=black, fill=green!30]
 \tikzstyle{arrow} = [thick,->,>=stealth]

%强调定义
\newcommand{\Emph}{\textbf}

%取消单词切断
\tolerance=1
\emergencystretch=\maxdimen
\hyphenpenalty=10000
\hbadness=10000


%随机段落,用于示例
\usepackage{lipsum}   