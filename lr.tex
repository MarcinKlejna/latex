 % !TEX TS-program = xelatexmk
 \documentclass[UTF8,10pt, twoside]{article} 
 \usepackage{ctexcap}   %中文支持 
 %页面纸张设置
 \usepackage{geometry}
 \geometry{a4paper, papersize={210mm,297mm}, scale=0.8} %A4
 \geometry{left=2cm,right=2cm,top=2cm,bottom=2cm}
 
 % 设置中文段落首行缩进 
 \usepackage{indentfirst}
 \setlength{\parindent}{2em }  
 \usepackage{titlesec}    
 \titleformat{\section}{\LARGE\bfseries}{\thesection}{1em}{}
 \usepackage{amsmath,amssymb}
 \usepackage{bm}
  
 
 \begin{titlepage}
 	\title{逻辑回归及其并行化}
 	\author{宫庆义\\gongqingyi@qq.com} 
 	\date{\today}
 \end{titlepage}
 
 \begin{document} 
 	\pagestyle{empty}
 	\maketitle                         %生成标题  
 	\thispagestyle{empty}
 	\newpage
   
\section{简介}
在计算广告学中, 通常使用逻辑回归(Logistic Regressions)来预估一个广告被点击的可能性,逻辑回归是对线性模型的一个归一化变换,其模型为:
\begin{equation}
z = p(y = 1|\bm{x}) = \frac{1}{1+e^{-\bm{w}\bm{x}}}
\end{equation}
这个模型的含义是,在给定一个广告的特征向量\(\bm{x}=(x_1,x_2,\ldots,x_n)\)的前提下,评估其被点击的可能性,这个模型也可以这样看待,对点击概率\(p\),非点击概率\(1-p\)之比(几率)的对数 \(\ln\frac{p}{1-p} = \bm{w}\bm{x} \)进行线性回归,  经过整理后,即可得上述逻辑回归模型。


\section{模型求解}
\subsection{目标函数}
对于这样一个模型,我们希望通过历史数据\(\bm{X}=(\bm{x}_1^T, \bm{x}_2^T, \ldots, \bm{x}_n^T)^T \),\(\bm{y}=(y_1, y_2, \ldots, y_n)^T\) 求出模型的参数\(\bm{w}\)。这里通常采用极大似然估计的方法,首先写出似然函数:
\begin{equation}
L=\prod_{n=1}^{N}\Big[p(y_n=1|\bm{x}_n)\Big]^{y_n}\Big[1-p(y_n=1|\bm{x}_n)\Big]^{1-y_n}
\end{equation}
这个问题可以转化成
\begin{equation}
 \min\limits_{\bm{w}} J = -\ln L + \alpha \|\bm{w}\|_1 + \beta \|\bm{w}\|_2^2
\end{equation}
 通常用梯度下降法(SGD)或者拟牛顿法(LBFGS),求出使得\(J\)最小的最优化参数\(\bm{w}\)。这里\(\|\bm{w}\|_1\),\(\|\bm{w}\|_2^2\)用来做模型选择和防止过拟合。
 
\subsection{目标函数梯度}
为了得到梯度,需要把目标函数\(J\)对参数\(\bm{w}\)求导:
\begin{equation}
\begin{aligned} 
\frac{-\partial \ln L}{\partial \bm{w}} &= -\sum_{n=1}^{N}\Big[  y_n \cdot \frac{1}{z_n} \cdot \frac{\partial z_n}{\partial \bm{w}}  -  (1-y_n) \cdot \frac{1}{1-z_n}\cdot \frac{\partial z_n}{\partial \bm{w}}  \Big]\\
&=\sum_{n=1}^N(z_n-y_n)\bm{x}_n
\end{aligned} 
\end{equation}

\begin{equation}
\frac{\partial \alpha \|\bm{w}\|_1}{ \partial w_i} = 
\begin{cases}
\begin{aligned} 
\alpha \qquad  w_{i} &> 0  \cr 
-\alpha \qquad  w_{i} &< 0  \cr 
\alpha \qquad  w_{i} &= 0, \quad \frac{-\partial \ln L}{\partial w_{i}}<-\alpha \cr 
-\alpha \qquad  w_{i} &= 0, \quad \frac{-\partial \ln L}{\partial w_{i}}>\alpha \cr
0 \qquad       w_{i} &= 0, \quad -\alpha\leqslant\frac{-\partial \ln L}{\partial w_{i}}\leqslant\alpha
\end{aligned}
\end{cases}
\end{equation}

\begin{equation}
\frac{\partial \beta \|\bm{w}\|_2^2}{ \partial \bm{w}} = 2 \beta \bm{w}
\end{equation}

\subsection{参数更新}
\begin{equation}
\bm{w}_{k+1} = \bm{w}_k - \eta \cdot \frac{\partial J}{\partial \bm{w}_k}
\end{equation}
  
  
\section{计算流程}
在具体的计算中,需要多次迭代来求出最优的\(\bm{w}\),每一次迭代主要分为两块,\(J\)的梯度计算,\(\bm{w}\)更新。数据的每一个样本\(\bm{x}\)是一个高维稀疏向量,样本是按行存放在不同机器节点上。假设有2台机器,各自存放了3条样本, 样本的维数是7维,第1台机器的3条样本为:
\[x_1 =  (1,1,1,0,1,0,0) \]
\[x_2 =  (0,0,1,1,0,0,0) \]
\[x_3 =  (0,1,1,0,0,0,0) \]

第2台机器的3条样本为:
\[x_4 = (1,0,0,0,1,0,1) \]
\[x_5 = (0,0,1,0,1,1,0) \]
\[x_6 = (1,0,1,0,1,0,0) \]

待求模型\(\bm{w}\)被切成2块(单机内存无法存放),第1,2,3,4维放在第1台机器,第5,6,7维放在第2台机器。
\newline\newline
我们发现,第1台机器的样本第6,7维全是0, 第2台机器的样本,第2,4维全是0。那么最终\(\bm{w}\)第6,7维的值由第2台机器的样本\(x_4,x_5\)计算得到,\(\bm{w}\)第2,4维的值由第1台机器的样本\(x_1,x_2,x_3 \)计算得到,而\(\bm{w}\)其他维(第1,3,5维)的值,由两台机器上的其他样本\(x_1,x_2,x_3 x_4, x_5, x_6\)计算得到。这说明,两台机器可以各自进行梯度的计算,和\(\bm{w}\)的更新,但是对于\(\bm{w}\)的第1,3,5维的更新,其对应的梯度,需要求和两台机器的在每一轮迭代过程中的梯度(第1,3,5维)作为全局梯度,然后每台机器用这个求和了的梯度,去更新\(\bm{w}\)的第1,3,5维。
\newline\newline
所以,在计算过程中,第2台机器需要发送梯度的第1,3维给第1台机器进行汇总,第1台机器需要发送梯度的第5维给第2台机器进行汇总,然后第1台机器,用汇总的梯度去更新\(\bm{w}\)的第1,3维,第2台机器用汇总的梯度去更新\(\bm{w}\)的第5维梯度。
\newline\newline
因此,更新\(\bm{w}\)的第1,3,5维需要两台机器先互相通信梯度数据,更新\(\bm{w}\)的第2,4维由第1台机器单独完成,更新\(\bm{w}\)的第6,7维,由第2台机器单独完成。
\newline\newline
如果样本数增多,特征维数增大,只需逐步往计算集群中添加机器,来分担梯度的计算,模型的更新,当然,集群中机器之间需要通信的数据量也会随之增大,在发送,接收数据的时候,还需要考虑网络传输的最优化,防止拥堵。

\section{基于并行矩阵库Petsc}
我们想只关注梯度的计算和\(\bm{w}\)的更新,即只负责代数计算的部分,而不去关注数据通信部分,另外梯度的计算,也希望用过矩阵库完成,
例如\(-\ln L\)对\(\bm{w}\)的偏导,可以把所有样本当成一个矩阵\(X\),这个偏导就是 \(X^T\left(\bm{z}-\bm{y}\right)\),而无需通过遍历样本1到N进行计算求和。

\end{document}
